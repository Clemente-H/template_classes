\section{Discusión: Implicancias y Reflexiones}

\begin{frame}[fragile]{Preguntas Técnicas para Discutir}
\begin{small}
  \textbf{1. ¿Llegamos al límite de escalar transformers?}
  \begin{itemize}
    \item Sora, Veo3 — ¿solo son modelos más grandes?
    \item ¿O hay innovaciones arquitectónicas que no conocemos?
  \end{itemize}

  \vspace{0.3cm}

  \textbf{2. ¿Los modelos entienden o solo memorizan?}
  \begin{itemize}
    \item ¿Realmente comprenden física del sonido?
    \item ¿O solo interpolan entre ejemplos vistos?
  \end{itemize}

  \vspace{0.3cm}

  \textbf{3. ¿Cuál es el rol de los datos vs la arquitectura?}
  \begin{itemize}
    \item Open-Sora vs Sora — ¿diferencia fundamental o solo escala?
  \end{itemize}
\end{small}
\end{frame}

\begin{frame}[fragile]{Preguntas Éticas para Discutir}
\begin{small}
  \textbf{1. ¿Cómo manejamos la autenticidad del contenido?}
  \begin{itemize}
    \item ¿Watermarking obligatorio?
    \item ¿Registros de proveniencia?
    \item ¿Confiamos en detección?
  \end{itemize}

  \vspace{0.3cm}

  \textbf{2. ¿Quién es responsable del mal uso?}
  \begin{itemize}
    \item ¿Los creadores de modelos?
    \item ¿Los usuarios?
    \item ¿Las plataformas que alojan contenido?
  \end{itemize}

  \vspace{0.3cm}

  \textbf{3. ¿Deberíamos watermarkear todo contenido generado?}
  \begin{itemize}
    \item Ventajas: trazabilidad, detección
    \item Desventajas: puede ser removido, falsos positivos
  \end{itemize}
\end{small}
\end{frame}

\begin{frame}[fragile]{Preguntas Sociales para Discutir}
\begin{small}
  \textbf{1. El futuro del trabajo creativo:}
  \begin{itemize}
    \item ¿Reemplaza o aumenta creatividad humana?
    \item ¿Qué pasa con artistas, músicos, editores de video?
  \end{itemize}

  \vspace{0.3cm}

  \textbf{2. Acceso democratizado vs concentración de poder:}
  \begin{itemize}
    \item Open-source vs modelos cerrados
    \item ¿Quién controla la narrativa?
  \end{itemize}

  \vspace{0.3cm}

  \textbf{3. ¿Cómo educamos a la sociedad?}
  \begin{itemize}
    \item Alfabetización mediática en era de contenido sintético
    \item ¿Cómo enseñamos escepticismo saludable?
  \end{itemize}
\end{small}
\end{frame}

% ============================================================================
% SECCIÓN: RECURSOS Y CIERRE
%============================================================================

