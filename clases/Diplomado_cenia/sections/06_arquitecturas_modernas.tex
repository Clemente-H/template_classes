\section{Arquitecturas Modernas: Cómo Resuelven Estos Problemas}

\begin{frame}[fragile]{La Evolución de las Soluciones}

  \textbf{Recordemos los problemas que identificamos:}
  \begin{itemize}
    \item Sincronización temporal audio-video
    \item Coherencia cross-modal
    \item Escalabilidad a videos largos
    \item Calidad en ambas modalidades
  \end{itemize}

  \vspace{0.3cm}

  \textbf{Veremos cómo los modelos modernos atacan estos problemas:}
  \begin{enumerate}
    \item \textbf{Open-Sora}: Transparencia en la arquitectura
    \item \textbf{Sistemas propietarios}: Sora, Veo3, Kling
    \item \textbf{Tendencias arquitectónicas comunes}
  \end{enumerate}
\end{frame}

\begin{frame}[fragile]{Open-Sora: ¿Por Qué Es Relevante?}
\begin{small}
  Paper: \url{https://arxiv.org/pdf/2412.20404}

  Repositorio: \url{https://github.com/hpcaitech/Open-Sora}

  \vspace{0.3cm}

  \textbf{¿Por qué estudiarlo?}
  \begin{itemize}
    \item \textbf{Es open-source}: Pueden experimentar y ver el código
    \item \textbf{Documenta las decisiones arquitectónicas} que los sistemas cerrados no revelan
    \item \textbf{Representa el estado del arte abierto}
  \end{itemize}

  \vspace{0.3cm}

  \textbf{Componentes clave:}
  \begin{itemize}
    \item \textbf{Diffusion Transformer (DiT)} para video
    \item \textbf{Integración de condicionamiento temporal}
    \item \textbf{Estrategias de entrenamiento multi-escala}
  \end{itemize}
\end{small}
\end{frame}

\begin{frame}[fragile]{Open-Sora: ¿Qué Resuelve de los Problemas Clásicos?}
\begin{small}
  \textbf{1. Consistencia temporal larga:}
  \begin{itemize}
    \item Usa \textbf{attention sobre secuencias} completas de frames
    \item No como RNN (VideoGPT) que procesa frame por frame
    \item Puede "ver" toda la secuencia a la vez
  \end{itemize}

  \vspace{0.3cm}

  \textbf{2. Escalabilidad:}
  \begin{itemize}
    \item \textbf{Entrenamiento progresivo}: Empieza con videos cortos/baja resolución
    \item Gradualmente aumenta complejidad
    \item Permite entrenar en hardware limitado
  \end{itemize}

  \vspace{0.3cm}

  \textbf{3. Eficiencia:}
  \begin{itemize}
    \item \textbf{Latent diffusion}: No trabaja en píxeles crudos
    \item Similar a Stable Diffusion (que vieron) pero para video
    \item Reduce costo computacional dramáticamente
  \end{itemize}
\end{small}
\end{frame}

\begin{frame}[fragile]{Sistemas Propietarios: Sora, Veo3, Kling}
\begin{small}
  \textbf{Lo que sabemos (y lo que no):}

  \vspace{0.3cm}

  \textbf{Arquitectura probable:}
  \begin{itemize}
    \item Probablemente \textbf{DiT-based} (Diffusion Transformers)
    \item Entrenamiento en datasets masivos propietarios
    \item \textbf{Generación de audio nativa} (no post-procesado)
  \end{itemize}

  \vspace{0.3cm}

  \textbf{La brecha open vs closed:}
  \begin{itemize}
    \item \textbf{Calidad de datos de entrenamiento}: Probablemente órdenes de magnitud más datos
    \item \textbf{Compute disponible}: Clusters masivos de GPUs
    \item \textbf{Técnicas de fine-tuning no publicadas}: "Secret sauce"
  \end{itemize}

  \vspace{0.3cm}

  \begin{alertblock}{La pregunta importante}
  ¿La diferencia es arquitectónica o es solo escala (datos + compute)?
  Probablemente: \textbf{mayormente escala}.
  \end{alertblock}
\end{small}
\end{frame}

\begin{frame}[fragile]{Tendencias Arquitectónicas Comunes}
\begin{small}
  Más allá de los modelos específicos, hay patrones claros:

  \textbf{1. Diffusion Transformers (DiT):}
  \begin{itemize}
    \item Reemplaza U-Net (de Stable Diffusion) con transformers puros
    \item Mejor escalabilidad y consistencia temporal
  \end{itemize}

  \vspace{0.2cm}

  \textbf{2. Latent representations:}
  \begin{itemize}
    \item Nadie modela píxeles directamente
    \item VAE o codecs neurales (como EnCodec para audio)
  \end{itemize}

  \vspace{0.2cm}

  \textbf{3. Multi-modal training:}
  \begin{itemize}
    \item Entrenamiento conjunto de audio y video desde el inicio
    \item No "pegar" dos modelos separados
  \end{itemize}

  \vspace{0.2cm}

  \textbf{4. Progressive training:}
  \begin{itemize}
    \item Empezar simple (corto, baja resolución)
    \item Gradualmente aumentar complejidad
  \end{itemize}
\end{small}
\end{frame}

% ============================================================================
% SECCIÓN: PROBLEMAS ABIERTOS
%============================================================================

