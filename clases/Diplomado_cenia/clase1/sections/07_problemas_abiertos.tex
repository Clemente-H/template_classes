\section{Problemas Abiertos y Fronteras de Investigación}

\begin{frame}[fragile]{Lo Que Aún No Está Resuelto}
\begin{small}
  A pesar del progreso, hay limitaciones fundamentales:

  \textbf{1. Sincronización fina:}
  \begin{itemize}
    \item Los labios aún se desincronizan en casos complejos
    \item Especialmente en conversaciones largas o lenguajes no-inglés
  \end{itemize}

  \textbf{2. Física del sonido:}
  \begin{itemize}
    \item Los modelos no entienden acústica real
    \item Reverberación, distancia, oclusión — todo es aprendido estadísticamente
    \item Fallan en escenarios fuera de distribución
  \end{itemize}

  \textbf{3. Consistencia de identidad:}
  \begin{itemize}
    \item En videos largos, los personajes "derivan"
    \item Cambios sutiles en apariencia frame a frame
  \end{itemize}

  \textbf{4. Edición coherente:}
  \begin{itemize}
    \item Modificar una modalidad (ej. cambiar audio) sin romper la otra
    \item Aún requiere regeneración completa
  \end{itemize}
\end{small}
\end{frame}

\begin{frame}[fragile]{Direcciones de Investigación Activas}
\begin{small}
  \textbf{1. World models audiovisuales:}
  \begin{itemize}
    \item No solo aprender correlaciones estadísticas
    \item \textbf{Entender física}: cómo los objetos suenan según material, tamaño, etc.
    \item Ejemplo: Genie 3 de Google (world model para robótica)
  \end{itemize}

  \textbf{2. Representaciones unificadas:}
  \begin{itemize}
    \item Un solo espacio latente para audio y video
    \item No dos ramas separadas que luego se "pegan"
  \end{itemize}

  \textbf{3. Generación interactiva:}
  \begin{itemize}
    \item Sistemas que respondan en tiempo real
    \item Edición iterativa (como Photoshop pero para video+audio)
  \end{itemize}

  \textbf{4. Control fino:}
  \begin{itemize}
    \item Más allá de texto: control espacial, temporal, semántico
    \item Interfaces que permitan especificar exactamente qué queremos
  \end{itemize}
\end{small}
\end{frame}

% ============================================================================
% SECCIÓN: DISCUSIÓN
%============================================================================

